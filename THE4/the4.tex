\documentclass[11pt]{article}
\usepackage[utf8]{inputenc}
\usepackage{float}
\usepackage{amsmath}
\usepackage{amssymb}

\usepackage[hmargin=3cm,vmargin=6.0cm]{geometry}

\usepackage{setspace}  
%\topmargin=0cm
\topmargin=-2cm
\addtolength{\textheight}{6.5cm}
\addtolength{\textwidth}{2.0cm}
%\setlength{\leftmargin}{-5cm}
\setlength{\oddsidemargin}{0.0cm}
\setlength{\evensidemargin}{0.0cm}

% symbol commands for the curious
\newcommand{\setZp}{\mathbb{Z}^+}
\newcommand{\setR}{\mathbb{R}}
\newcommand{\calT}{\mathcal{T}}

\begin{document}

\section*{Student Information } 
%Write your full name and id number between the colon and newline
%Put one empty space character after colon and before newline
Full Name : Emre Berk Kaya \\
Id Number : 2380590 \\
\doublespacing
% Write your answers below the section tags
\section*{Answer 1}

    We can select 1 star from 10 stars by using $C(10,1)$.\\
    We can select 8 non-habitable planets from 80 non-habitable planets by using $C(80,8)$. \\
    We can select 2 habitable planets from 20 habitable planets by using $C(20,2)$. \\
    By Product Rule, we can make all of this selections by 
    \begin{gather*}
        C(10,1).C(80,8).C(20,2)
    \end{gather*} 
    Let the habitable planets be denoted by $H$ and non-habitable planets denoted by $n$ \\
    After the selection, we can consider the ordering. There are 3 cases:
    \begin{enumerate}
        \item There are 6 non-habitable planets between 2 habitable planets:\\
            \hspace*{5mm} We can consider 6 non-habitable planets and 2 habitable planets as a block:\\
            \hspace*{10mm} $n-[H-n-n-n-n-n-n-H]-n$\\
            \hspace*{5mm} To form this block, we will choose 6 non-habitable planets from existing 8. \\
            \hspace*{5mm} We can make this selection by $C(8,6)$. These 6 non-habitable planets can be ordered by $6!$.\\
            \hspace*{5mm} Two habitable planets can ordered by $2!$ .\\
            \hspace*{5mm} Remaining 2 non-habitable planets and the block can be ordered by $3!$\\
            \hspace*{5mm} By product rule, whole forming can be done by
            \begin{gather*}
                C(8,6).6!.2!.3!
            \end{gather*}
        \item There are 7 non-habitable planets between 2 habitable planets: \\
            \hspace*{5mm} We can consider 7 non-habitable planets and 2 habitable planets as a block:\\
            \hspace*{10mm} $n-[H-n-n-n-n-n-n-n-H]$\\
            \hspace*{5mm} To form this block, we will choose 7 non-habitable planets from existing 8. \\
            \hspace*{5mm} We can make this selection by $C(8,7)$. These 7 non-habitable planets can be ordered by $7!$.\\
            \hspace*{5mm} Two habitable planets can ordered by $2!$ .\\
            \hspace*{5mm} Remaining 1 non-habitable planet and the block can be ordered by $2!$\\
            \hspace*{5mm} By product rule, whole forming can be done by
            \begin{gather*}
                C(8,7).7!.2!.2!
            \end{gather*}
        \item There are 8 non-habitable planets between 2 habitable planets:\\
            \hspace*{5mm} For this time, there is nothing but the block:\\
            \hspace*{10mm} $[H-n-n-n-n-n-n-n-H]$\\
            \hspace*{5mm} We chose all of non-habitable planets with C(8,8).\\
            \hspace*{5mm} We can order these 8 non-habitable planets by $8!$\\
            \hspace*{5mm} Two habitable planets can ordered by $2!$ .\\
            \hspace*{5mm} By product rule, whole forming can be done by
            \begin{gather*}
                C(8,8).8!.2!
            \end{gather*}
    \end{enumerate}
        
        
    By Sum Rule and Product Rule, there are
    \begin{gather*}
        C(10,1).C(80,8).C(20,2). \Big[C(8,6).6!.2!.3! + C(8,7).7!.2!.2! + C(8,8).8!.2! \Big]
    \end{gather*}
    ways of forming the galaxy.
    
\newpage

\section*{Answer 2}

    First we can arrange the terms: 
        \begin{align*}
            a_n -2a_{n-1} -15a_{n-2} +36a_{n-3} = 2^n
        \end{align*}
    We can find homogeneous solutions with using characteristic equation:
        \begin{align*}
            \lambda^3-2\lambda^2-15\lambda+36 = 0\\
            (\lambda-3)^2(\lambda+4)=0
        \end{align*}
    Then, our characteristic roots are $\lambda_{1}=3$ with multiplicity 2, and $\lambda_2 = -4$. The homogeneous solution is
        \begin{align*}
            a_n^{(h)} = (An+B)3^n + C(-4)^n \-\ \-\  A,B,C \in \mathbb{R}
        \end{align*}
    As the non-homogeneity term is $2^n$, we can propose the particular solution
        \begin{align*}
            a_n^{(p)} = D.2^n \-\ \-\ D \in \mathbb{R}
        \end{align*}
    If we substitute particular solution in the recurrence relation,
        \begin{gather*}
            D2^n -2D2^{n-1} -15D2^{n-2}+36D2^{n-3} = 2^n\\
            6D2^{n-3} = 2^n\\
            D = \frac{4}{3}
        \end{gather*}
    Then, our particular solution becomes
        \begin{gather*}
            a_n^{(p)} = \frac{4}{3}.2^n = \frac{2^{n+2}}{3}
        \end{gather*}
    The general solution is
        \begin{gather*}
            a_n = a_n^{(h)} + a_n^{(p)}\\
            a_n = (An+B)3^n + C(-4)^n + \frac{2^{n+2}}{3} \hspace{2mm} A,B,C \in \mathbb{R}
        \end{gather*}


\section*{Answer 3}

    We have 5 odd and 5 even digits on $[0,9]$. The number of n-digit valid codes will be denoted by $a_n$. \\
    If the code is 1 digit long, this digit must be an odd number. We can select it with $C(5,1)$ and there is only one ordering. Therefore, $a_1 = 5$\\
    Suppose that we have n-1 digit code and we will add one more digit to this code. We have two cases:
    \begin{enumerate}
        \item If this n-1 digit code contains odd number of odd digits (i.e. it is a valid code), we can denote the number of these codes by $a_{n-1}$. We must add an even digit to keep this code valid. We can select the even digit by using $C(5,1)$. Then, by Product Rule, the number of n-digit codes in this case is
        \begin{gather*}
            C(5,1).a_{n-1} = 5a_{n-1}
        \end{gather*}
        \item If this n-1 digit code contains even number of odd digits (i.e. it is not a valid code), we can denote the number of these codes by $10^{n-1}-a_{n-1}$ as there are total $10^{n-1}$ n-1 digit codes and $a_{n-1}$ valid n-1 digit codes. We must add an odd number to keep this code valid. We can select the odd digit by using $C(5,1)$. Then, by Product Rule, the number of n-digit codes in this case is 
        \begin{gather*}
            C(5,1) \big[ 10^{n-1}-a_{n-1} \big] = 5.10^{n-1}-5a_{n-1}
        \end{gather*} 
    \end{enumerate}
    
    Then, by Sum Rule, total number of n-digit valid codes is 
    \begin{gather*}
        a_n = 5a_{n-1} + 5.10^{n-1}-5a_{n-1}  = 5.10^{n-1}
    \end{gather*}
    
    
\section*{Answer 4}
    Let $G(k) =\sum_{k=0}^{\infty} a_k x^k$ be our generating function.
    We can write this equation as
    \begin{gather*}
        \sum_{k=3}^{\infty} a_k x^k = 3x \sum_{k=3}^{\infty} a_{k-1}x^{k-1} -3x^2 \sum_{k=3}^{\infty} a_{k-2}x^{k-2} + x^3 \sum_{k=3}^{\infty} a_{k-3}x^{k-3}
    \end{gather*}
    This equation is equivalent to
    \begin{gather*}
        G(x) -a_0-a_1x-a_2x^2 = 3x \big[G(x)-a_0-a_1x \big] -3x^2 \big[G(x)-a_0 \big] + x^3 G(x)\\
        G(x)-1-3x-6x^2 = 3x G(x)-3x-9x^2-3x^2G(x)+3x^2+x^3G(x)\\
        (-x^3+3x^2-3x+1)G(x) = 1 \\
        G(x) = \dfrac{1}{(1-x)^3}
    \end{gather*}
    From Chapter 8.4 Table 1 Line 9, we can write this function as 
    \begin{gather*}
       \dfrac{1}{(1-x)^3} =  \sum_{k=0}^{\infty} C(k+2,k)x^k 
    \end{gather*}
    Then we can conclude that
    \begin{gather*}
        a_k = C(k+2,k) = \dfrac{(k+2)(k+1)}{2} = \dfrac{k^2+3k+2}{2}
    \end{gather*}
    

\section*{Answer 5}
\paragraph{a.}

    Let $ ((a, b),(c, d)) \in R$ and $a,b,c,d \in \mathbb{Z^+}$. Then,  $a+d = b+c$.
\begin{itemize}
    \item $((a,b),(a,b)) \in R$ as $a+b = a+b$. Hence, $R$ is reflexive.
    
    \item $((c,d),(a,b))\in R$ since we know that $c+b = d+a$. Hence, $R$ is symmetric.
    
    \item Suppose that $((c,d),(e,f)) \in R$, $e,f \in \mathbb{Z^+}$. Then, $c+f = d+e$.
    \begin{itemize}
        \item We can derive $a-b = c-d$ since $ ((a, b),(c, d))\in R$.
        \item Also we can derive $c-d = e-f$ since $((c,d),(e,f)) \in R$.
        \item Then, we can conclude that
        \begin{gather*}
            a-b = c-d = e-f\\
            a-b = e-f\\
            a+f = b+e
        \end{gather*}
            Then, $((a,b),(e,f)) \in R$. Hence, $R$ is transitive.
    \end{itemize}
    Since $R$ is reflexive, symmetric, transitive, $R$ is an equivalence relation.
\end{itemize}
    
    
    
    
\paragraph{b.}

    The equivalence class of (1,2) with respect to $R$ contains all $(x,y)$ such that $((1,2),(x,y)) \in R$, $x,y \in \mathbb{Z^+}$. Therefore, $1+y = 2+x$. Then, $y-x = 1$. It can be represented as
    \begin{gather*}
         \big[ (1,2) \big]_R = \big\{(x,y) \-\ | \-\ y-x = 1, \-\ x,y \in \mathbb{Z^+} \big\}
    \end{gather*}
    

\end{document}