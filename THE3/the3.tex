\documentclass[12pt]{article}
\usepackage[utf8]{inputenc}
\usepackage[dvips]{graphicx}
\usepackage{epsfig}
\usepackage{fancybox}
\usepackage{verbatim}
\usepackage{array}
\usepackage{latexsym}
\usepackage{alltt}
\usepackage{float}
\usepackage{amsmath}

\usepackage{setspace}
\usepackage{amssymb}

\usepackage{hyperref}
\usepackage{listings}
\usepackage{color}
\usepackage[hmargin=3cm,vmargin=5.0cm]{geometry}
\topmargin=-1.8cm
\addtolength{\textheight}{6.5cm}
\addtolength{\textwidth}{2.0cm}
\setlength{\oddsidemargin}{0.0cm}
\setlength{\evensidemargin}{0.0cm}

\newcommand{\HRule}{\rule{\linewidth}{1mm}}
\newcommand{\kutu}[2]{\framebox[#1mm]{\rule[-2mm]{0mm}{#2mm}}}
\newcommand{\gap}{ \\[1mm] }
\newcommand{\Mod}[1]{\ (\mathrm{mod}\ #1)}


\newcommand{\Q}{\raisebox{1.7pt}{$\scriptstyle\bigcirc$}}

\lstset{
    %backgroundcolor=\color{lbcolor},
    tabsize=2,
    language=C++,
    basicstyle=\footnotesize,
    numberstyle=\footnotesize,
    aboveskip={0.0\baselineskip},
    belowskip={0.0\baselineskip},
    columns=fixed,
    showstringspaces=false,
    breaklines=true,
    prebreak=\raisebox{0ex}[0ex][0ex]{\ensuremath{\hookleftarrow}},
    %frame=single,
    showtabs=false,
    showspaces=false,
    showstringspaces=false,
    identifierstyle=\ttfamily,
    keywordstyle=\color[rgb]{0,0,1},
    commentstyle=\color[rgb]{0.133,0.545,0.133},
    stringstyle=\color[rgb]{0.627,0.126,0.941},
}


\begin{document}



\noindent 
\HRule \\[3mm]
\small
\begin{tabular}[b]{lp{3.8cm}r}
{} Middle East Technical University &  &
{} Department of Computer Engineering \\
\end{tabular} \\
\begin{center}

                 \LARGE \textbf{CENG 223} \\[4mm]
                 \Large Discrete Computational Structures \\[4mm]
                \normalsize Fall '2020-2021 \\
                    \Large Homework 3 \\
                \normalsize Student Name and Surname: Emre Berk Kaya \\
                \normalsize Student Number: 2380590 \\
\end{center}
\HRule

\doublespacing

\section*{Question 1}

    By Fermat's Little Theorem, $2^{11} \equiv 4^{11} \equiv 6^{11} \equiv 1 \Mod{11}$. \\
    We can conclude that\\
    $(2^{11})^{2} \equiv 2^{22} \equiv 1^2 \equiv 1 \Mod{11}$\\
    $(4^{11})^4 \equiv 4^{44} \equiv 1^4 \equiv 1 \Mod{11}$\\
    $(6^{11})^6 \equiv 6^{66} \equiv 1^6 \equiv 1 \Mod{11}$\\
    $(10^{11})^10 \equiv 10^{110} \equiv 1^{10} \equiv 1 \Mod{11}$\\
    $(8^{11})^7 \equiv 8^{77} \equiv 1^7 \equiv 1 \Mod{11}$, Then, \\
    $8^{80} \equiv 8^{77}.8^3 \equiv 1^7.6 \equiv 6 \Mod{11}$. Hence,\\
    $2^{22} + 4^{44} + 6^{66} + 10^{110} + 8^{80} \equiv 1+1+1+1+6 \equiv 10 \Mod{11}$
    
    

\section*{Question 2}
    
    $(7n+4) = (5n+3) + (2n+1)$\\
    $(5n+3) = 2(2n+1) + (n+1) $\\
    $(2n+1) = (n+1) + n$ \\
    $(n+1) = 1(n) + 1  $\\
    Hence, by Euclid's Algorithm, $gcd(5n+3, 7n+4) = 1.$


\section*{Question 3}
    
    $m^2 = n^2 + kx$\\
    $m^2 - n^2 = kx$\\
    $(m-n)(m+n) = kx$\\
    $\frac{(m-n)(m+n)}{x} = k$, $k \in \mathbb{Z}$. Therefore, $x|(m-n)(m+n)$ by the definition of divisibility.\\
    If $x | (m-n)$, the statement holds.\\
    Suppose that $x \not| (m-n)$.
    Then, gcd(x,(m-n)) = 1. By Bezout's Identity,\\
    $xa + (m-n)b = 1$ for some $a,b \in \mathbb{Z}$. \\
    $xa(m+n) + (m-n)(m+n)b = (m+n)$\\
    $xa(m+n) + kxb = (m+n)$\\
    $x ( a(m+n)+kb) = (m+n)$. Then $x|(m+n)$ and the statement holds.\\
    Hence if x is prime, $x|(m+n)$ or $x|(m-n)$.
    
    
    
    

\section*{Question 4}
    
    Let $P(n) = \sum_{i=1}^{n}(3i-2) = \frac{n(3n-1)}{2}.$\\
    $P(1)$ holds as $\frac{1(3.1-1)}{2} = 1$.\\
    Let k be a positive integer and $P(k)$ holds.\\
    Then $P(k+1)$ should be equal to $ P(k) + 3(k+1)-2$.\\
    $P(k) + 3(k+1)-2 = \frac{k(3k-1)}{2} + 3(k+1)-2$.\\
    $= \frac{3k^2-k+6k+6-4}{2}$\\
    $= \frac{(3k+2)(k+1)}{2}$\\
    $= \frac{(k+1)(3(k+1)-1)}{2}$\\
    $= P(k+1)$.\\
    Hence, by Mathematical Induction, $1+4+7+...+(3n-2) = \frac{n(3n-1)}{2}$.
    
    

\end{document}

